\documentclass{article}
\usepackage[utf8]{inputenc}
\usepackage{tabularx}
\usepackage[a4paper,left=.01cm,right=1cm,top=1cm,bottom=1cm,landscape]{geometry}

\usepackage{xfrac}

\begin{document}
\small
	\begin{tabular}{c|l}
        Avocadosalat & $\sfrac{1}{2}$ \textbf{Gurke} würfeln, mit $\sfrac{1}{2}$ TL \textbf{Salz} 20' rasten; gewürfelte \textbf{Avocado}, 200g \textbf{Shrimps}, 1 Prise \textbf{Zucker}, \textbf{Pfeffer}, 2EL \textbf{Olivenöl} dazu; 15' rasten\\\hline

        Brocc e Dofu & leicht gebr. \textbf{Brocc.} in 1 Tasse 1:1 \textbf{Salz}-\textbf{Zucker}-Wasser dünsten; \textbf{Sojas.}, Wasser, \textbf{Stärke}, \textbf{Knobi}, \textbf{Ingwer} in Pfanne andicken; in Stärke gewälzten \textbf{Dofu} braten \\\hline

        Butter Dofu {\scriptsize(3P)}&  2EL \textbf{Stärke}, \textbf{Salz}, \textbf{Pfeffer} \& 1EL \textbf{Öl} auf 400g abgetupften zerissenen \textbf{Tofu}, 20' 175°C; \textbf{Öl}, 1 \textbf{Zw.}, 2 \textbf{Knobis}, \textbf{Ingwer} in Pfanne, (jetzt \textbf{Reis}), 2TL (\textbf{Garam M.} + \\&
        \textbf{Kreuzk.} + \textbf{gem. Kor.}), \textbf{Chillip.} dazu, ziehen, 2EL \textbf{Tomatenm.}, $\sfrac{1}{2}$l \textbf{Passata}, \textbf{Butter} dazu, 5'-10' köcheln, $\sfrac{1}{4}$l \textbf{Kokosnussm.}, Spritzer \textbf{Zitrone}, TL \textbf{Zucker}, \textbf{Salz}\\\hline

        Burger & \textbf{Burgerbuns}, \textbf{Bulette}nersatz, \textbf{Salat}, \textbf{Sauce}, \textbf{Tomaten}, \textbf{Champs}; \textit{Pommes}: \textbf{Kartoffeln}, \textbf{Salz}, \textbf{Öl}\\\hline

        Chili con \textit{Carne} & große Dose \textbf{passierte Tomaten}, Dose \textbf{Bohnen}, Dose \textbf{Mais}, 400g \textbf{Rindsfaschiertes}, \textbf{Zwiebel}, \textbf{Mehl} \\\hline
		
        Egg Bene // \textit{Flore} & 1 \textbf{Ei} in \textbf{Essig}wasser pochieren, 1 Bl. gebratenen \textbf{Serrano} // \textit{\textbf{Blattspinat} + \textbf{Knobi} + \textbf{Muskatn.}}, 40ml \textbf{Hollandaise}, $\sfrac{1}{2}$ \textbf{Toastie}\\\hline

		Gnocchi mit Sauce & \textbf{Gnocchi}; \textit{Sauce}: \textbf{Zwiebel}, \textbf{Champignons} zerkleinert anrösten, mit \textbf{Weißwein} ablöschen, \textbf{Suppe/Wasser}, \textbf{Sahne}, angebratene \textbf{Süßkartoffel} dazu; köcheln\\\hline
		
        Linsensuppe {\scriptsize(3P)}& 2 gepr. \textbf{Knobis} \& 1 \textbf{Zw.} in 1EL \textbf{Butter} glasig, $\sfrac{1}{2}$TL \textbf{Kurkuma} + 1TL (\textbf{Garam M.} + \textbf{Curry} + \textbf{Kreuzk.}) dazu, 800g \textbf{Dosentom.} abtropfen, schneiden und mit 180g\\&
        \textbf{rot. Linsen}, 2TL \textbf{Zitronensaft}, 600ml \textbf{Gmiasbriah}, 300ml \textbf{Kokosnussm.} dazu und aufkochen; ohne Deckel 25' köcheln\\\hline

        K.J.'s Instants & \textbf{Instantnudeln} kochen, in Pfanne mit \textbf{Ei}, \textbf{Knoblauchpulver}, Instantpulver, \textbf{Butter} anbraten; \textbf{Lauchzw.}\\\hline
		
        $\mathcal{M}$'s Schupfnudeln&\textbf{Zw.}, \textbf{Champs}, $\sfrac{1}{2}$kg \textbf{Fasch.} br.; TL \textbf{Pappul.}, EL \textbf{TMark.}, 150g (\textbf{Schlag}+\textbf{Crème F.}) in gef. Form mit \textbf{Schupfn.}, \textbf{Sauerkr.} schichten;
        ger. \textbf{Käs}, \textbf{But.} drauf; 180°C 45'
		\\\hline

        Melanzani Bharta {\scriptsize(4P)} & 2 gr. \textbf{Mel.} in 1.5cm 220°C 30' - zerkl.; in 3EL \textbf{Öl} kurz TL \textbf{Kor.samen} br., 1 Bd. \textbf{Frühl.zw.} \& \textbf{r. Zw.} fein geh. dazu bis glasig, 4 \textbf{Knobis} reinpr., 1’ br.; 2 \textbf{Tom.} fein,\\&
        400g \textbf{geh. Dosent.} 2' dazu, kälter; 2EL \textbf{gem. Kor.}, TL \textbf{gem. Kreuzk.}, TL \textbf{Chili}, $\sfrac{1}{2}$TL \textbf{Kurkuma}, Pr. \textbf{gem. Nelken} dazu, rühren bis Aroma entf.; Mel. dazu, \textbf{Salz}\\\hline
		
        Melanzanipfanne & \textbf{Melanzani}, \textbf{Zwiebel}, \textbf{Walnüsse}, \textbf{Paprikapulver}, \textbf{Tomatensauce} (irgendeiner Art) \\\hline

		Mie-Nudelpfanne&\textbf{Mie-Nudeln} kochen; \textbf{Zucchini, Paprika, Champignons, ..} und \textbf{Shrimps} braten und dazu, \textbf{Soyas.} drauf; \textbf{Lauchzw.} ganz am Ende\\\hline
		
        Pasta Pomodoro & fruchtigste \textbf{Tomaten} vierteln und in \textbf{Öl} anbraten bis sie zerfallen, \textbf{Knoblauch} in Scheiben, Nudelwasser und \textbf{Agavendicksaft} dazu \\\hline
        
        Pasta Spinacchi & \\\hline

        Pasta Aglio e Olio & \textbf{Knoblauch} gefeinscheibt in \textbf{Öl} tanzen lassen; \textbf{Parmesan} ehrlich gesagt keine Ahnung wie das Rezept geht\\\hline

        Pasta Lax e Avo & \textbf{Avocado} mit etwas Nudelwasser zermantschgern, \textbf{Salz}, \textbf{Pfeffer}; \textbf{Lax} knusprig braten\\\hline

        Pasta Lemone & Schale einer $\sfrac{1}{2}$ \textbf{Bio Zitrone} in Streifen in \textbf{Olivenöl} sautieren; \textbf{Nudeln} samt \textbf{Butter}st., Nudelwasser, Zitronensaft dazu und erhitzen; kurz abkühlen; \textbf{Parmesan} rein\\\hline

        Pizzabaguette & \textbf{Aufbackweckerl}, \textbf{Tomatenmark}, \textbf{Mozzarella}, \textbf{Belag} nach Wahl\\\hline

        Ramen {\scriptsize(1P)} & $\sfrac{1}{2}$ \textbf{Knobi}, 1 \textbf{getr. Tom.} anbraten, 1EL \textbf{Sojas.}, $\sfrac{1}{2}$EL \textbf{Majo} dazu; \textbf{Ei} kochen; \textbf{Mie N.} in \textbf{Gemüsebr.} mit \textbf{Brocc.} (\& \textbf{Pak Choi}, ...) kochen; alles zsm.; \textbf{Noribl.}\\\hline

        Raiberdatschi {\scriptsize(3P)}& 1.2kg \textbf{Kartoffeln}, 1 \textbf{rote Zw.}, \textbf{Pfeffer}, \textbf{Muskatn.}, (\textbf{Ei}), 3EL \textbf{Mehl} mit ordentlich \textbf{Salz} zermantschgern, in heiße Pfanne als Datschis mit viel \textbf{Öl} von bd. Seiten braten\\\hline

        Reispapieromlette & \textbf{Reispapier} in mittelwarme Pfanne; \textbf{Sesamöl}, \textbf{Ei} drauf und verqirlen; heiß; \textbf{Gewürz}, \textbf{Tomate}, \textbf{Schmelzkäse}, \textbf{Zwiebel}, \textbf{Lauchzw.} dazu; 2. Reisp. drauf; bis knusprig\\\hline

        Risotto & 300g \textbf{Risottor.}, \textbf{rote Zw.} in \textbf{Öl} glasig sautieren; mit \textbf{Weißw.} löschen; 1l \textbf{Gemüsebrühe} "dazu"; \textbf{getr. Tom.}, \textbf{Shrimps}, \textbf{Parmi} kurz vor Ende dazu; \textbf{Ruccola}\\\hline
		
		S. K. Toast&\textbf{Toast}, \textbf{Schinken}, \textbf{Käse}, \textbf{Ketchup}, \textbf{Mayo}\\\hline		
		
		Salat&\textbf{Rucola}, \textbf{Feldsalat}, \textbf{Blattsalat}, \textbf{Tomaten}, gebratene \textbf{Champignons}, \textbf{Kartoffelsalat} \quad\textit{Dressing}: (\textbf{Öl,Essig,Salz,Zucker}), (\textbf{Honey,Mustard,Öl,Essig})\\\hline
		
        \textbf{Schupfnudel}pfanne & \textit{entweder} \textbf{Gemüse (Champs, Süßkart., Zucchini, ..)} \quad\textit{oder} (\textbf{Sauerkraut}, \textbf{Speck}) \quad\textit{oder} süß (\textbf{Mohn} , \textbf{Staubzucker}) \textit{oder} (\textbf{Apfelmus}, geriebene \textbf{Nüsse})\\\hline

        Spaghetti $\mathcal{C}$& 200g \textbf{Spaghetti} kochen, \textbf{Speck} und \textbf{Knobi} braten; Nudeln, \textbf{Parmesan}, \textbf{Pfeffer}, 2 verquirlte \textbf{Eier} (nicht stocken lassen!) dazu; Nudelwasser nach Bedarf\\\hline

        Sushi {\scriptsize(3P)} & 1 \textbf{Gurke, Avocado}, 150g \textbf{Lax}, 5 \textbf{Noribl.}; 350g \textbf{Sushir.} waschen, 30' in Wasser rasten, in 0.5l aufkochen, 15' mit Deckel köcheln, 20' betucht rasten, in fl. Schüssel\\&
        auflockern, \textit{Sushi-Zu} (in 70ml \textbf{Reisessig} 10g \textbf{Salz}, 35g \textbf{Zucker} bei schw. Hitze auflösen, abkühlen lassen) unter Fächerung einarbeiten, 30' feucht bedeckt rasten lassen\\\hline

        Sommerrollen & \textbf{Reispapier}, \textbf{Glasn.}, \textbf{Shrimps}, \textbf{Karotte}, \textbf{Gurke}, \textbf{Lauchzw.}; \textit{Sauce:} 3EL \textbf{Sojas.}, 2EL \textbf{Erdnussb.}, 1EL \textbf{Honigartiges}, 1 \textbf{Knobi}zehe, bissi \textbf{Ingwer} \& \textbf{Zitrone} \\\hline

        Wrap & \textbf{Wraps}, \textbf{Fleisch}ersatz, \textbf{Salat}, \textbf{Sauce}, \textbf{Käse}, \textbf{Gemüse}  \\\hline

        \\\hline
        \\\hline
        \\\hline
        \\\hline\hline
%Süßes-----------------------------------------------------
        Crumbleschüsserl {\scriptsize(3P)}& 80g \textbf{Mehl}, 50g \textbf{braunen Zucker}, $\sfrac{1}{2}$ Packerl \textbf{Vanillezucker}, 50g \textbf{But.} mit Prise \textbf{Salz} vermengen und kneten bis crumbelig; auf beliebige \textbf{Früchte} streußeln; 20' 180°C
        \\\hline

        $\mathcal{F}$'s Mom's Cookies & 280g \textbf{Schok.} Wasserbad; 280g \textbf{Mehl}, TL \textbf{Backp.}, \textbf{Salz} mischen; 90g w. \textbf{But.} cremig rühren, 200g \textbf{Zuck.}, 1P \textbf{VZuck.}, 3 \textbf{Eier} dazu, Rest dazu; \textbf{ges. Cashew}; 12' 160°C
        \\\hline

        Kaiserschmarrn & \\\hline
		
        Palatschinken&	\textbf{Mehl}, \textbf{Eier}, \textbf{Milch}, Prise \textbf{Salz}, \textbf{Zucker}, \textbf{Butter}, \textbf{Marmelade}, \textbf{Nutella}, Früchte (\textbf{Banane}, \textbf{Erdbeere}, ..)\\\hline

        Pancakes & 300ml \textbf{Naturjoghurt}, 150g \textbf{Mehl}, 1 Pkg. \textbf{Backpulver}, Prise \textbf{Salz} und 2 EL \textbf{Zucker} 3 verquirlten \textbf{Eiern} unterrühren; \textbf{Ahornsirup}, \textbf{Butter}, \textbf{Früchte}, \textbf{Schokolade}\\\hline

        Tassenküchlein & 6 EL \textbf{Mehl}, 3 EL \textbf{Kakao}, $\sfrac{1}{4}$ TL \textbf{Backp.}, (1 EL \textbf{Zucker}), Prise \textbf{Salz} verrühren; 6 EL \textbf{Milch}, 3 EL \textbf{Öl}, 1 gscheider EL \textbf{Nutella} verrühren; knappe 2' bei 700W in Mikro
        \\\hline



        \\\hline
        \\\hline

	\end{tabular}
\end{document}
